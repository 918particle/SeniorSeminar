\title{Syllabus for Senior Seminar in Physics (PHYS499)}
\author{Dr. Jordan Hanson - Whittier College Dept. of Physics and Astronomy}
\date{\today}
\documentclass[10pt]{article}
\usepackage[margin=1.5cm]{geometry}
\usepackage{outlines}
\usepackage{hyperref}
\usepackage{graphicx}
\begin{document}
\maketitle

\begin{abstract}
PHYS 499 represents the capstone course in the physics major.  This course is designed to develop essential skills in scientific research and communication. Throughout the course, students will conduct significant literature research, write a comprehensive paper, and give three progressively challenging presentations on their chosen topic. These presentations will cover fundamental physics concepts, explore peer-reviewed journal articles, and culminate in a final talk based on the final paper.  In addition, the course will help students explore career and post-graduate opportunities in physics. They will reflect on their future career or educational paths, gain insights into potential options, and receive guidance on applying for jobs and graduate programs, with support from the Career Center. 
\end{abstract}
\noindent
\textit{\textbf{Pre-requisites}: Declaration of major in physics, along with senior status.} \\
\textit{\textbf{Course credits, Liberal Arts Categorization}: 3 Credits, Life Lab 3, Written Communication 3} \\
\textit{\textbf{Regular course hours and location}: Tuesday and Thursday, 3:30 - 4:30 pm, SLC 232.} \\
\textit{\textbf{Instructor contact information}: Discord: 918particle, Email: jhanson2@whittier.edu, Office: SLC 222.} \\
\noindent
\textit{\textbf{Office hours}: Book online appointments: \url{https://calendar.app.google/svuevm9JH4tG3CXg6}.} \\
\textit{\textbf{Attendance/Absence}: Students must give scientific presentations based on literature reviews on specific dates.  However, if exigent circumstances arise, and students need to reschedule, this should be resolved beforehand with the instructor.} \\ 
\textit{\textbf{Late work policy}: Late work is generally not accepted, but is left to the discretion of the instructor.} \\
\textit{\textbf{Text}: No course textbook will be used, but rather open-access scientific journal articles in physics, engineering, mathematics, and computer science will be used.} \\
\textit{\textbf{Course Work and Grading}: The course grade will be a weighted average of assignment scores.  See Tab. \ref{tab:grades}.}
\begin{table}
\centering
\begin{tabular}{| c | c | l |}
\hline
\textbf{Assignment} & \textbf{Weight} & \textbf{Date} \\ \hline
Class participation & 10 \% & Completed during class\\ \hline
First literature review & 15 \% & February 13th, 2026 \\ \hline
First presentation & 15 \% & February 20th, 2026 \\ \hline
Second literature review & 15 \% & March 20th, 2026 \\ \hline
Second presentation & 15 \% & March 27th, 2026 \\ \hline
Third literature review & 15 \% & April 30th, 2026 \\ \hline
Third presentation & 15 \% & May 5th, 2026 \\ \hline
\end{tabular}
\caption{\label{tab:grades} Grade weighting. The final project presentation may be given via \textbf{Option A} or \textbf{Option B} (see below).}
\end{table} \\
\noindent
\textit{\textbf{Grade Settings}: $\geq 60\%, <70\%$ = D, $\geq 70\%, <80\%$ = C, $\geq 80\%, <90\%$ = B, $\geq 90\%, <100\%$ = A. Pluses and minuses: 0-3\% minus, 3\%-6\% straight, 6\%-10\% plus (e.g. 79\% = C+, 91\% = A-).} \\
\textit{\textbf{ADA Statement on Disability Services}: Whittier College is committed to make learning experiences as accessible as possible. If you experience physical or academic barriers due to a disability, you are encouraged to contact Student Disability Services (SDS) to discuss options. To learn more about academic accommodations, email disabilityservices@whittier.edu, call (562) 907-4825, or go to SDS which is located on the ground floor of Wardman Library.} \\
\textit{\textbf{Academic Honesty:} \url{https://www.whittier.edu/policies/academic/honesty}} \\
\noindent
\textit{\textbf{Course Objectives}:}
\begin{itemize}
\item Performing a literature review on a significant topic of interest within physics and engineering
\item Scientific writing at the professional level, wit logical flow, and citations
\item Scientific communication at the professional level
\begin{enumerate}
    \item Scientific speaking with a live audience
    \item Preparing visual tools: graphics, tables, and illustrations
\end{enumerate}
\item Professional career exploration and preparation
\end{itemize}
\clearpage
\noindent
\textbf{Literature review.} Students will learn to research a topic by locating peer-reviewed, scientific books and articles.  Based on pre-determined criteria, a bibliography is compiled that addresses the scientific topic or question.  Insights are gained by a critical and thorough reading of the works. \\ \\
\noindent
\textbf{Grading criteria:}
\begin{enumerate}
    \item Physical understanding of the topic or question, based on the resources located and knowledge of physics ... 50 \%
    \item Ability to answer questions posed about the topic using the materials gained through the search ... 50 \%
\end{enumerate} \vspace{1cm}
\textbf{Scientific presentation.} Students will create a scientifically accurate, visually engaging, and logically comprehensible summary of the topic of the literature review, meant to be given in person. \\ \\
\noindent
\textbf{Grading criteria:}
\begin{enumerate}
    \item Clear communication of technical ideas and processes ... 25 \%
    \item Logical flow of the arguments and ideas ... 25 \%
    \item Creating technically sound figures and tables ... 25 \%
    \item Ability to answer questions from the intended audience ... 25 \%
\end{enumerate} \vspace{1cm}
\textbf{Special topics regarding career exploration and preparation.}  The following topics will be presented during class sessions throughout the semester:
\begin{itemize}
    \item Three sectors within STEM: academia and education, government service, and the private sector
    \item Categories of engineering work: acquisitions and design, prototyping and startups, development and production, testing and verification, reliability and life-cycle analysis
    \item Forms of government service and private sector participation:
    \begin{enumerate}
        \item \textit{The defense sector}: large and small firms, public-private partnerships, security clearances, and classified work
        \item \textit{The space sector}: NASA facilities and private corporations
    \end{enumerate}
    \item Academia: colleges and universities, K-12 education, and the national laboratory system
    \item Private sector participation: engineering and physics roles in energy, software and computing engineering, AI, defense and aerospace, electrical and electronics engineering, robotics and mechanical engineering, biomedical and bioengineering, environmental engineering, automotive and transport
    \item Using job search tools and social media to classify, search, and apply for job roles
    \begin{enumerate}
        \item LinkedIn
        \item ClearanceJobs
        \item Handshake
        \item Physics Today Jobs
    \end{enumerate}
\end{itemize}
\end{document}