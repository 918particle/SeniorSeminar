\documentclass[10.5pt]{article}
\usepackage[margin=1.5cm]{geometry}
\usepackage{amsmath}
\usepackage{hyperref}
\usepackage{enumitem}

\title{Using arXiv.org for a Physics Literature Review}
\author{}
\date{}

\begin{document}
\maketitle

\section{What is arXiv?}

\textbf{arXiv.org} is an open-access preprint repository where researchers post
manuscripts prior to (or alongside) peer-reviewed journal publication.
In physics, arXiv is often the \emph{primary} source for the most recent research,
with papers appearing months before journal publication.

arXiv is organized into subject categories such as:
\begin{itemize}
    \item \texttt{physics.class-ph} (Classical Physics)
    \item \texttt{physics.optics}
    \item \texttt{cond-mat.*} (Condensed Matter Physics)
    \item \texttt{hep-*} (High Energy Physics)
    \item \texttt{astro-ph.*} (Astrophysics)
\end{itemize}

\section{Finding Relevant Papers}

\subsection{Keyword Search}

Use the search bar at \url{https://arxiv.org} with:
\begin{itemize}
    \item Technical keywords (e.g.\ ``photonic crystal band gap'')
    \item Author names for known researchers
    \item arXiv identifiers (e.g.\ \texttt{arXiv:2301.01234})
\end{itemize}

For more precise searches, use the \textbf{Advanced Search} to:
\begin{itemize}
    \item Restrict subject categories
    \item Search titles vs.\ abstracts
    \item Filter by date range
\end{itemize}

\subsection{Browsing by Category}

Browsing recent submissions in a category is useful for:
\begin{itemize}
    \item Identifying current research trends
    \item Discovering new authors and groups
    \item Staying updated during an active project
\end{itemize}

Daily or weekly email alerts can be enabled for specific categories.

\section{Evaluating arXiv Papers}

Since arXiv papers are not guaranteed to be peer-reviewed, evaluation is essential:

\begin{itemize}
    \item Check if the paper is later published in a journal
    \item Look for well-known authors or institutions
    \item Examine citation counts using Google Scholar
    \item Assess clarity, mathematical rigor, and references
\end{itemize}

arXiv versions often improve over time; always check the \emph{latest version}.

\section{Using arXiv for a Literature Review}

A typical workflow:
\begin{enumerate}[label=\arabic*.]
    \item Start with review papers or highly cited preprints
    \item Follow reference lists to earlier foundational work
    \item Track forward citations using Google Scholar
    \item Organize papers by topic, method, or result
\end{enumerate}

arXiv is especially useful for:
\begin{itemize}
    \item Rapidly evolving fields
    \item Computational or theoretical physics
    \item Learning standard notation and models
\end{itemize}

\section{Citing arXiv Papers}

arXiv papers can be cited directly in Bib\TeX. Example:

\begin{verbatim}
@article{Smith2023arXiv,
  title={Title of the Paper},
  author={Smith, John and Doe, Jane},
  journal={arXiv preprint arXiv:2301.01234},
  year={2023}
}
\end{verbatim}

If a journal version exists, cite the published paper instead.

\section{Limitations and Best Practices}

\begin{itemize}
    \item arXiv does not replace peer-reviewed journals
    \item Not all physics subfields use arXiv equally
    \item Always cross-check important results
\end{itemize}

Best practice is to use arXiv alongside:
\begin{itemize}
    \item Peer-reviewed journals
    \item Review articles
    \item Conference proceedings
\end{itemize}

\section{Conclusion}

arXiv is an essential tool for modern physics research, enabling rapid access
to cutting-edge work. When used critically and systematically, it is a powerful
resource for conducting a high-quality physics literature review.

\end{document}
